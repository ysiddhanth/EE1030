\documentclass[journal]{IEEEtran}
\usepackage[a5paper, margin=10mm]{geometry}
%\usepackage{lmodern} % Ensure lmodern is loaded for pdflatex
\usepackage{tfrupee} % Include tfrupee package


\setlength{\headheight}{1cm} % Set the height of the header box
\setlength{\headsep}{0mm}     % Set the distance between the header box and the top of the text


%\usepackage[a5paper, top=10mm, bottom=10mm, left=10mm, right=10mm]{geometry}

%
\usepackage{gvv-book}
\usepackage{gvv}
\setlength{\intextsep}{10pt} % Space between text and floats

\makeindex

\begin{document}
\bibliographystyle{IEEEtran}
\onecolumn
\title{
%	\logo{
GATE ASSIGNMENT 1

\large{EE1030 : Matrix Theory}

Indian Institute of Technology Hyderabad
%	}
}
\author{Yellanki Siddhanth

(EE24BTECH11059)
}	


%code by ysiddhanth


\maketitle





\bigskip

\renewcommand{\thefigure}{\theenumi}
\renewcommand{\thetable}{\theenumi}
 
    
        \textbf{2007 CE 52 to 68}
    
    \begin{enumerate}
    \item{
          	A triangular open channel has a vertex angle of 90$^\circ$ and carries flow at a critical depth of 0.30m. The discharge in the channel is:\ \text{  }\hfill
                
            \begin{multicols}{4}
				\begin{enumerate}
					\item $0.08 \text{ m}^3/\text{s}$ 
					
					\item $0.11 \text{ m}^3/\text{s}$ 
					
					\item $0.15 \text{ m}^3/\text{s}$ 
					
					\item $0.2 \text{ m}^3/\text{s}$
				\end{enumerate}
			\end{multicols}
            }
    %code by ysiddhanth 
    \item{
            Flow rate of a fluid (density = $1000 kg/m^3$) in a small diameter tube is $800 \, \text{m}^3/\text{s}$. The length and the diameter of the tube are $2 \, \text{m}$ and $0.5 \, \text{mm}$, respectively. The pressure drop in $2 \, \text{m}$ length is equal to $2.0 \, \text{MPa}$. The viscosity of the fluid is\hfill
                
            \begin{multicols}{4}
                \begin{enumerate}
                    \item $0.025 \, {N.s}/{m^2}$
                    \item $0.012 \, {N.s}/{m^2}$
                    \item $0.0092 \, {N.s}/{m^2}$
                    \item $0.00102 \, {N.s}/{m^2}$
                \end{enumerate}
            \end{multicols}
        }
\item{
        	
        	The flow rate in a wide rectangular open channel is $2.0 \, \text{m}^3/\text{s}$ per meter width. The channel bed slope is 0.002. The Manning’s roughness coefficient is 0.012. The slope of the channel is classified as
        	\hfill
        	
        	\begin{multicols}{4}
        		\begin{enumerate}
        			\item Critical  
        			\item Horizontal  
        			\item Mild  
        			\item Steep  
        		\end{enumerate}
        	\end{multicols}
        	
        }
    \item{
     
           The culturable command area for a distributary channel is 20,000 hectares. Wheat grown in the entire area and the intensity of irrigation is 50\%. The kor period for wheat is 30 days and the kor water depth is 120mm. The outlet discharge for the distributary should be:\hfill
                
            \begin{multicols}{4}
                \begin{enumerate}
                	\item $2.85 \text{ m}^3/\text{s}$ 
                	
                	\item $3.21 \text{ m}^3/\text{s}$ 
                	
                	\item $4.63 \text{ m}^3/\text{s}$ 
                	
                	\item $2.85 \text{ m}^3/\text{s}$
                \end{enumerate}
            \end{multicols}
        
        }
 	\item{
        	An isolated 4-hour storm occurred over a catchment as follows:
        	
        	\begin{center}
        		\begin{tabular}{|c|c|c|c|c|}
        			\hline
        			Time & 1st hour & 2nd hour & 3rd hour & 4th hour \\
        			\hline
        			Rainfall (mm) & 9 & 28 & 12 & 7 \\
        			\hline
        		\end{tabular}
        	\end{center}
        	
        	The $\phi$ index for the catchment is 10 mm/h. The estimated runoff depth from the catchment due to the above storm is\text{ }
        	\hfill
        	
        	
        	\begin{multicols}{4}
        		\begin{enumerate}
        			\item 10mm
        			\item 16mm
        			\item 20mm
        			\item 23mm
        		\end{enumerate}
        	\end{multicols}
        	
        }
 	\item{
			Two electrostatic precipitators (ESPs) are in series. The fractional efficiencies of the upstream and downstream ESPs for size $d_p$ are 80\% and 65\%, respectively. What is the overall efficiency of the system for the same $d_p$?
			\hfill
			
			
			\begin{multicols}{4}
				\begin{enumerate}
					\item 100\%
					\item 93\%
					\item 80\%
					\item 65\%
				\end{enumerate}
			\end{multicols}
			
		}
 	\item{
			50g of CO$_2$ and 25g of CH$_4$ are produced from the decomposition of municipal solid waste (MSW) with a formula weight of 120g. What is the average per capita green house gas production in a city of 1 million people with a MSW production rate of 500 ton / day?\text{ }
			\hfill
			
			
			\begin{multicols}{2}
				\begin{enumerate}
					\item 104 g/day 
					\item 120 g/day 
					\item 208 g/day 
					\item 313 g/day 
				\end{enumerate}
			\end{multicols}
			
		}
    \item{
            The extra widening required for a two-lane national highway at a horizontal curve of 300 m radius, considering a wheel base of 8m and a design speed of 100kmph is\text{ }
             \hfill
                
            \begin{multicols}{4}
                \begin{enumerate}
                	\item 0.42m
                	\item 0.62m
                	\item 0.82m
                	\item 0.92m
                \end{enumerate}
            \end{multicols}

        %code by ysiddhanth 
        
        }
    \item{
            While designing a hill road with a ruling gradient of 6\%, if a sharp horizontal curve of 50m radius is encountered, the compensated gradient at the curve as per the Indian Roads Congress specifications should be
             \hfill
                
            \begin{multicols}{4}
                \begin{enumerate}
                	\item 4.4\%
                	\item 4.75\%
                	\item 5.0\%
                	\item 5.25\%
                \end{enumerate}
            \end{multicols}
        
        }
    \item{
            The design speed on a road is 60 km/h. Assuming the driver reaction time of 2.5 seconds and coefficient of friction of pavement surface as 0.35, the required stopped distance for two-way traffic on a single lane road is 
             \hfill
                
			\begin{multicols}{4}
				\begin{enumerate}
					\item 82.1m
					\item 102.4
					\item 164.2m
					\item 186.4m
				\end{enumerate}
			\end{multicols}
        
        }
    \item{
        
           	The width of the expansion joint is 20mm in a cement concrete pavement. The laying temperature is $20^{\circ}$C and the maximum slab temperature in summer is $60^{\circ}$C. The coefficient of thermal expansion of concrete is $10 \times 10^{-6} mm/mm/^{\circ}C$ and the joint filler compresses up to 50\% of the thickness. The spacing between expansion joints should be
             \text{   }\hfill
                
            \begin{multicols}{4}
                \begin{enumerate}
                	\item 20m
                	\item 25m
                	\item 30m
                	\item 40m
                \end{enumerate}
            \end{multicols}
        
        }
    \item{
	
			The following data pertains to the number of commercial vehicles per day for the design of a flexible pavement for a national highway as per IRC: 37-1984 
			
			\begin{center}
				\begin{tabular}{|c|c|c|}
					\hline
					Type of commercial vehicle & Number of vehicles per day & Vehicle Damage Factor \\
					\hline
					Two axle trucks & 2000 & 5 \\
					Tandem axle trucks & 200 & 6 \\
					\hline
				\end{tabular}
			\end{center}
			
			Assuming a traffic growth factor of 7.5\% per annum for both types of vehicles, the cumulative number of standard axle load repetitions (in million) for a design life of ten years is:
			\text{   }\hfill
			
			\begin{multicols}{4}
				\begin{enumerate}
					\item 44.6
					\item 57.8
					\item 62.4
					\item 78.7
				\end{enumerate}
			\end{multicols}
			
		}
    \item{
	
			Match the following tests on aggregate and its properties:
			\begin{center}
				\begin{tabular}{|c|l|c|l|}
					\hline
					\textbf{Test} & \textbf{Property}  \\ \hline
					P. Crushing Test & 1. Hardness \\ \hline
					Q. Los Angeles abrasion test & 2. Weathering \\ \hline
					R. Soundness test & 3. Shape \\ \hline
					S. Angularity test & 4. Strength \\ \hline
				\end{tabular}
			\end{center}
			\text{   }\hfill
			
			\begin{multicols}{2}
				\begin{enumerate}
					\item  P-2, Q-1, R-4, S-3  
					
					\item  P-4, Q-2, R-3, S-1  
					 
					\item  P-3, Q-2, R-1, S-4  
					
					\item  P-4, Q-1, R-2, S-2  
				\end{enumerate}
			\end{multicols}
			
		}
    \item{
        
           The plan of a map was photo copied to a reduced size such that a line originally $100$mm, measures $90$mm. The original scale of the plan was $1:1000$. The revised scale is 
             \hfill
              
              \begin{multicols}{4}
              	\begin{enumerate}
              		\item 1:900
              		\item 1:11111
              		\item 1:1121
              		\item 1:1221
              	\end{enumerate}
              \end{multicols}
        
        }
        \item{
        	
        	The following table gives data of consecutive coordinates in respect of a closed theodolite traverse PQRSP.
        	
        	\begin{center}
        		\begin{tabular}{|c|c|c|c|c|}
        			\hline
        			\textbf{Station} & \textbf{Northing, m} & \textbf{Southing, m} & \textbf{Easting, m} & \textbf{Westing, m} \\ \hline
        			P & 400.75 & & & 300.5 \\ \hline
        			Q & 100.25 & & 199.25 & \\ \hline
        			R & & 199.0 & 399.75 & \\ \hline
        			S & & 300.0 & & 200.5 \\ \hline
        		\end{tabular}
        	\end{center}
        	
        	The magnitude and direction of error of closure in whole circle bearing are
        	
        	\hfill
        	
        	\begin{multicols}{4}
        		\begin{enumerate}
        			\item $2.0m$ and $450$
        			\item $2.82m$ and $3150$
        			\item $2.0m$ and $3150$
        			\item $3.42m$ and $450$
        		\end{enumerate}
        	\end{multicols}
        	
        }
            \item{
        	
        	The following measurements were made during testing a leveling instrument.
        	
        	\begin{center}
        		\begin{tabular}{|c|c|c|}
        			\hline
        			\textbf{Instrument at} & \multicolumn{2}{c|}{\textbf{Staff Reading at}} \\ \hline
        			& \(P_1\) & \(Q_1\) \\ \hline
        			P & 2.800 m & 1.700 m \\ \hline
        			Q & 2.700 m & 1.800 m \\ \hline
        		\end{tabular}
        	\end{center}
        	
        	\(P_1\) is close to P and \(Q_1\) is close to Q. If the reduced level of station P is 100.000 m, the reduced level of station Q is
        	
        	\hfill
        	
        	\begin{multicols}{4}
        		\begin{enumerate}
        			\item 99.000m
        			\item 100.000m
        			\item 101.000m
        			\item 102.000m
        		\end{enumerate}
        	\end{multicols}
        	
        }    
        \item{
        
        Two straight lines intersect at an angle of $60^\circ$. The radius of a curve joining the two straight lines is $600m$. The length of long chord and mid-ordinates in metres of the curve are
        \hfill
        
        \begin{multicols}{4}
        	\begin{enumerate}
        		\item 80.4, 600.00
        		\item 600.0, 80.4
        		\item 600.0, 39.89
        		\item 40, 89,300
        	\end{enumerate}
        \end{multicols}
        
        }
    \end{enumerate}
\end{document}



