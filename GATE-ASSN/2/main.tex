\documentclass[journal]{IEEEtran}
\usepackage[a5paper, margin=10mm]{geometry}
%\usepackage{lmodern} % Ensure lmodern is loaded for pdflatex
\usepackage{tfrupee} % Include tfrupee package


\setlength{\headheight}{1cm} % Set the height of the header box
\setlength{\headsep}{0mm}     % Set the distance between the header box and the top of the text


%\usepackage[a5paper, top=10mm, bottom=10mm, left=10mm, right=10mm]{geometry}

%
\usepackage{gvv-book}
\usepackage{gvv}
\setlength{\intextsep}{10pt} % Space between text and floats

\makeindex

\begin{document}
\bibliographystyle{IEEEtran}
\onecolumn
\title{
%	\logo{
GATE ASSIGNMENT 2

\large{EE1030 : Matrix Theory}

Indian Institute of Technology Hyderabad
%	}
}
\author{Yellanki Siddhanth

(EE24BTECH11059)
}	


%code by ysiddhanth


\maketitle





\bigskip

\renewcommand{\thefigure}{\theenumi}
\renewcommand{\thetable}{\theenumi}
 
    
        \textbf{2012 MA 53 to 65}\\
\begin{enumerate}
\item[]{
	    \textbf{LINKED QUESTION AND ANSWERS}\\
\textbf{Statement for Linked Answer Questions 52 and 53:} \\
The joint probability density function of two random variables $X$ and $Y$ is given as
\[
f(x,y) = 
\begin{cases}
	\frac{6}{5}\brak{x+y^2}, & 0\leq x\leq 1,0\leq y\leq 1 \\
	0, & \text{elsewhere}
\end{cases}
\]}
    %code by ysiddhanth 
	\item{
		$E\brak{X} $
		and
		$ E\brak{Y} $
		are, respectively,\text{  }\hfill
		
		\begin{multicols}{4}
			\begin{enumerate}
				\item $\frac{2}{5} \text{ and } \frac{3}{5}$
				
				\item $\frac{3}{5} \text{ and } \frac{3}{5}$
				
				\item $\frac{3}{5} \text{ and } \frac{6}{5}$
				
				\item $\frac{4}{5} \text{ and } \frac{6}{5}$
			\end{enumerate}
		\end{multicols}
	}
   	\item{
    	$Cov\brak{X,Y}$ is: \text{  }\hfill
    	
    	\begin{multicols}{4}
    		\begin{enumerate}
    			\item -0.01 
    			
    			\item 0
    			
    			\item 0.01 
    			
    			\item 0.02
    		\end{enumerate}
    	\end{multicols}
    }
\item[]{
 \textbf{Statement for Linked Answer Questions 54 and 55:}\\
 Consider the functions
 $
 f(z) = \frac{z^2 + \alpha z}{\brak{z+1}^2}
 $
 and
 $
 g(z) = \sinh(z - \frac{{\pi}}{{2\alpha}}),
 $
 where $\alpha \neq 0$.}
    \item{
            The residue of $f(z)$ at its pole is equal to 1. Then the value of $\alpha$ is.
                
            \begin{multicols}{4}
                \begin{enumerate}
                    \item -1
                    \item 1
                    \item 2
                    \item 3
                \end{enumerate}
            \end{multicols}
        }
	\item{
        	
        	For the value of $\alpha$ obtained in Q.54, the function $g\brak{z}$ is not conformal at a point.
        	\hfill
        	
        	\begin{multicols}{4}
        		\begin{enumerate}
        			\item $\frac{\pi\brak{1+3i}}{6}$
        			\item $\frac{\pi\brak{3+i}}{6}$  
        			\item $\frac{2\pi}{3}$
        			\item $\frac{i\pi}{2}$
        		\end{enumerate}
        	\end{multicols}
      }
%\end{enumerate}
\item[]{
\textbf{General Aptitude (GA) Questions (Compulsory)}

\textbf{Q. 56 – Q. 60 carry one mark each.}\\}
%\begin{enumerate}[resume]
        \item {Choose the most appropriate word from the options given below to complete the following
    sentence: \\ \textbf{
    Given the seriousness of the situation that he had to face, his \_\_\_ was impressive.} \\ 
    \begin{multicols}{4}
    	\begin{enumerate}
    		\item beggary 
    		\item nomenclature
    		\item jealousy
    		\item nonchalance
    	\end{enumerate}
    \end{multicols}}
    
    \item {Choose the most appropriate alternative from the options given below to complete the following
    sentence: \\ \textbf{
    If the tired soldier wanted to lie down, he \_\_\_ the mattress out on the balcony.}
    \begin{multicols}{4}
    	\begin{enumerate}
    		\item should take 
    		\item shall take
    		\item should have take
    		\item will have taken
    	\end{enumerate}
    \end{multicols}}    
    \item {If $\brak{1.001}^{1259} = 3.52$ and $\brak{1.001}^{2062} = 7.85$, then $\brak{1.001}^{3321} =$
    	\begin{multicols}{4}
	    	\begin{enumerate}
	    		\item 2.23 
	    		\item 4.33 
	    		\item 11.37 
	    		\item 27.64
	    	\end{enumerate}
    	\end{multicols}}    
    \item {One of the parts $\brak{A, B, C, D}$ in the sentence given below contains an ERROR. Which one of the
    following is INCORRECT? \\ 
    \textbf{I requested that he should be given the driving test today instead of tomorrow.}
        \begin{multicols}{2}
	    	\begin{enumerate}
				\item requested that
				\item should be given
				\item the driving test
				\item instead of tomorrow
	    	\end{enumerate}
    	\end{multicols}
	}
    \item {Which one of the following options is the closest in meaning to the word given below?\\
    \textbf{Latitude}
    	\begin{multicols}{4}
    		\begin{enumerate}
    			\item Eligibility
    			\item Freedom
    			\item Coercion
    			\item Meticulousness
    		\end{enumerate}
    	\end{multicols}
    
	}
	\item[]{
		\textbf{Q. 61 - Q. 65 carry two marks each.}
	}
    \item{
            There are eight bags of rice looking alike, seven of which have equal weight and one is slightly
            heavier. The weighing balance is of unlimited capacity. Using this balance, the minimum number
            of weighings required to identify the heavier bag is
             \hfill
                
            \begin{multicols}{4}
                \begin{enumerate}
                	\item 2
                	\item 3
                	\item 4
                	\item 8
                \end{enumerate}
            \end{multicols}

        %code by ysiddhanth 
        
        }
    \item{
            Raju has 14 currency notes in his pocket consisting of only Rs. 20 notes and Rs. 10 notes. The total
            money value of the notes is Rs. 230. The number of Rs. 10 notes that Raju has is
                
            \begin{multicols}{4}
                \begin{enumerate}
                	\item 5
                	\item 6
                	\item 9
                	\item 10
                \end{enumerate}
            \end{multicols}
        
        }
    \item{
            One of the legacies of the Roman legions was discipline. In the legions, military law prevailed
            and discipline was brutal. Discipline on the battlefield kept units obedient, intact and fighting,
            even when the odds and conditions were against them.\\
            \textbf{Which one of the following statements best sums up the meaning of the above passage?}
             \hfill
            
				\begin{enumerate}
					\item	Thorough regimentation was the main reason for the efficiency of the Roman legions even in
					adverse circumstances.
					\item The legions were treated inhumanly as if the men were animals.
					\item Discipline was the armies’ inheritance from their seniors.
					\item The harsh discipline to which the legions were subjected to led to the odds and conditions being
					against them.
				\end{enumerate}
        
        }
    \item{
        
           	A and B are friends. They decide to meet between 1 PM and 2 PM on a given day. There is a
           	condition that whoever arrives first will not wait for the other for more than 15 minutes. The
           	probability that they will meet on that day is
             \text{   }\hfill
                
			
			\begin{multicols}{4}
				\begin{enumerate}
					\item $\frac{1}{4}$
					\item $\frac{1}{16}$
					\item $\frac{7}{16}$
					\item $\frac{9}{16}$
				\end{enumerate}
			\end{multicols}
        
        }
    \item{
	
			The data given in the following table summarizes the monthly budget of an average household. \\
			\begin{tabular}[12pt]{ |c| c| c|}
    \hline
    \textbf{Variable} & \textbf{Description} & \textbf{Value}\\
	\hline
	$\vec{n}$ &Normal of Directrix& $\myvec{1 \\ -1} $\\
	\hline
	$\vec{c}$ & c of Directrix& $ 3$\\
	\hline
	$\vec{e}$ & Eccentricity of conic & $\frac{1}{2}$\\
	\hline
	$\vec{F}$ & Focus of conic &  $\myvec{1 \\ -1}$ \\
	\hline
\end{tabular}
 The approximate percentage of the monthly budget \textbf{NOT} spent on savings is
			\text{   }\hfill
			
			\begin{multicols}{2}
				\begin{enumerate}
					\item 10\%
					
					\item 14\%
					 
					\item 81\%
					
					\item 86\%
				\end{enumerate}
			\end{multicols}
			
		}
  
    \end{enumerate}
\end{document}



