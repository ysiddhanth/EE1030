\documentclass[journal]{IEEEtran}
\usepackage[a5paper, margin=10mm]{geometry}
%\usepackage{lmodern} % Ensure lmodern is loaded for pdflatex
\usepackage{tfrupee} % Include tfrupee package


\setlength{\headheight}{1cm} % Set the height of the header box
\setlength{\headsep}{0mm}     % Set the distance between the header box and the top of the text


%\usepackage[a5paper, top=10mm, bottom=10mm, left=10mm, right=10mm]{geometry}

%
\usepackage{gvv-book}
\usepackage{gvv}
\setlength{\intextsep}{10pt} % Space between text and floats

\makeindex

\begin{document}
\bibliographystyle{IEEEtran}
\onecolumn
\title{
%	\logo{
JEE ASSIGNMENT 3

\large{EE1030 : Matrix Theory}

Indian Institute of Technology Hyderabad
%	}
}
\author{Yellanki Siddhanth

(EE24BTECH11059)
}	


%code by ysiddhanth


\maketitle





\bigskip

\renewcommand{\thefigure}{\theenumi}
\renewcommand{\thetable}{\theenumi}
 
    
        \textbf{2021 Feb 25 Shift 2 1 to 15}
    
    \begin{enumerate}
    \item{
          	Let the system of linear equations
          	$4x + \lambda y + 2z = 0,
          	2x - y + z = 0,
          	\mu x + 2y + 3z = 0, \lambda, \mu \in \mathbb{R}$
          	Has a non-trivial solution. Then which of the following is true?\\ \text{  }\hfill
                {(2021 - 4 Marks)}
                \begin{multicols}{2}
					\begin{enumerate}
						\item $\mu = 6, \lambda \in \mathbb{R}$
						\item $\lambda = 2, \mu \in \mathbb{R}$
						\item $\lambda = 3, \mu \in \mathbb{R}$
						\item $\mu = -6, \lambda \in \mathbb{R}$
					\end{enumerate}
				\end{multicols}
            }
    %code by ysiddhanth 
    \item{
           	A pole stands vertically inside a triangular park $ABC$. Let the angle of elevation of the top of the pole from each corner of the park be $\frac{\pi}{3}$. If the radius of the circumcircle of $\triangle ABC$ is $2$, then the height of the pole is equal to
                \begin{multicols}{4}
                	\begin{enumerate}
                			\item $\frac{1}{\sqrt{3}}$
                			
                			\item $\sqrt{3}$
                			
                			\item $2\sqrt{3}$
                			
                			\item $\frac{2\sqrt{3}}{3}$
                	\end{enumerate}
                \end{multicols}
        }
\item{
        	
        	Let in a series of $2n$ observations, half of them are equal to $a$ and the remaining half are equal to $-a$. Also by adding a constant $b$ in each of these observations, the mean and standard deviation of the new set become $5$ and $20$, respectively. Then the value of $a^2 + b^2$ is equal to:
        	\hfill
        	{(2021 - 4 Marks)}
        	\begin{multicols}{4}
        		\begin{enumerate}
        			\item 250
        			\item 925
        			\item 650
        			\item 425
        		\end{enumerate}
        	\end{multicols}
        	
        }
    \item{
     
            Let $g\brak{x} = \int_{0}^{x} f\brak{t} dt$ where $f$ is a continuous function in $\sbrak{0, 3}$ such that $\frac{1}{3} \leq f\brak{t} \leq 1$ for all $t \in \sbrak{0, 1}$ and $0 \leq f\brak{t}$ for all $t \in \lbrak1, 3\rsbrak{}$. The largest possible interval in which $g\brak{3}$ lies is:\hfill
                {(2021 - 4 Marks)}
            \begin{multicols}{4}
                \begin{enumerate}
                    \item $\sbrak{1, 3}$
                    
                    \item $\sbrak{-1, -\frac{1}{2}}$
                    
                    \item $\sbrak{-\frac{3}{2}, -1}$
                    
                    \item $\sbrak{\frac{1}{3}, 2}$
                \end{enumerate}
            \end{multicols}
        
        }
    \item{
            If $15 \sin^4 \theta + 10 \cos^4 \theta = 6$, for some $\theta \in \mathbb{R}$, then the value of $27 \sec^6 \theta + 8 \csc^6 \theta$ is equal to:
           	\hfill
                {(2021 - 4 Marks)}
            
            \begin{multicols}{4}
				\begin{enumerate}
					\item 250
					\item 400
					\item 500
					\item 350
				\end{enumerate}
			\end{multicols}
        
        }
 	\item{
        	Let $f: \mathbb{R} - \cbrak{3} \rightarrow \mathbb{R} - \cbrak{1}$ be defined by $f\brak{x} = \frac{x - 2}{x - 3}$. Let $g: \mathbb{R} \rightarrow \mathbb{R}$ be given as $g\brak{x} = 2x - 3$. Then, the sum of all the values of $x$ for which $f^{-1}\brak{x}+ g^{-1}\brak{x} = \frac{13}{2}$ is equal to\\ \text{ }
        	\hfill
        	{(2021 - 4 Marks)}
        	
        	\begin{multicols}{4}
        		\begin{enumerate}
					\item 7
					
					\item 5
					
					\item 2
					
					\item 3
        		\end{enumerate}
        	\end{multicols}
        	
        }
 	\item{
			Let $S_1$ be the sum of the first $2n$ terms of an arithmetic progression. Let $S_2$ be the sum of the first $4n$ terms of the same arithmetic progression. If $\brak{S_2 - S_1}$ is $1000$, then the sum of the first $6n$ terms of the arithmetic progression is equal to:\\ \text{ }
			\hfill
			{(2021 - 4 Marks)}
			
			\begin{multicols}{4}
				\begin{enumerate}
					\item 3000
					\item 7000
					\item 5000
					\item 1000
				\end{enumerate}
			\end{multicols}
			
		}
 	\item{
			Let $S_1 = x^2 + y^2 = 9$ and $S_2 = \brak{x - 2}^2 + y^2 = 1$. Then the locus of the centre of a variable circle $S$ which touches $S_1$ internally and $S_2$ externally always passes through the points:\\ \text{ }
			\hfill
			{(2021 - 4 Marks)}
			
			\begin{multicols}{4}
				\begin{enumerate}
					\item $\brak{ \frac{1}{2}, \pm \frac{\sqrt{5}}{2}}$
					\item $\brak{ 2, \pm \frac{3}{2}}$
					\item $\brak{ 1, \pm 2 }$
					\item $\brak{ 0, \pm \sqrt{3}}$
				\end{enumerate}
			\end{multicols}
			
		}
    \item{
            Let the centroid of an equilateral triangle $ABC$ be at the origin. Let one of the sides of the equilateral triangle be along the straight line $x + y = 3$. If $R$ and $r$ be the radius of circumcircle and incircle respectively of $\Delta ABC$, then $\brak{R + r}$ is equal to\\ \text{ }
             \hfill
                {(2021 - 4 Marks)}
            \begin{multicols}{4}
                \begin{enumerate}
                    \item $2\sqrt{2}$    
                    \item $3\sqrt{2}$    
                    \item $7\sqrt{2}$    
                    \item $\frac{9}{\sqrt{2}}$
                \end{enumerate}
            \end{multicols}

        %code by ysiddhanth 
        
        }
    \item{
            In a triangle $ABC$, if vector $\overrightarrow{BC} = 8$, $\overrightarrow{CA} = 7$, $\overrightarrow{AB} = 10$, then the projection of the vector $AB$ on $AC$ is equal to:
             \hfill
                {(2021 - 4 Marks)}
            \begin{multicols}{4}
                \begin{enumerate}
                	\item $\frac{25}{4}$ 
                	\item $\frac{85}{14}$ 
                	\item $\frac{127}{20}$ 
                	\item $\frac{115}{16}$
                \end{enumerate}
            \end{multicols}
        
        }
    \item{
            Let in a Binomial distribution, consisting of 5 independent trials, probabilities of exactly 1 and 2 successes be 0.4096 and 0.2048 respectively. Then the probability of getting exactly 3 successes is equal to:
             \hfill
                {(2021 - 4 Marks)}
			\begin{multicols}{4}
				\begin{enumerate}
					\item $\frac{80}{243}$
					\item $\frac{32}{625}$
					\item $\frac{128}{625}$
					\item $\frac{40}{243}$
				\end{enumerate}
			\end{multicols}
        
        }
    \item{
        	Let $a$ and $b$ be two non-zero vectors perpendicular to each other and $|a| = |b|$. If $|a \times b| = |a|$, then the angle between the vectors $\brak{a + b + \brak{a \times b}}$ and $a$ is equal to:\\
             \text{   }\hfill
                {(2021 - 4 Marks)}
				\begin{multicols}{4}
	                \begin{enumerate}
	                   	\item $\sin^{-1}\brak{\frac{1}{\sqrt{3}}}$
	                   	\item $\cos^{-1}\brak{\frac{1}{\sqrt{3}}}$
	                   	\item $\sin^{-1}\brak{\frac{1}{\sqrt{6}}}$
	                   	\item $\cos^{-1}\brak{\frac{1}{\sqrt{2}}}$
	                \end{enumerate}
				\end{multicols}
        
        }
    \item{
	
			Let a complex number be $w = 1 - \sqrt{3}i$. Let another complex number $z$ be such that $|zw| = 1$ and $\arg\brak{z} - \arg\brak{w} = \frac{\pi}{2}$. Then the area of the triangle with vertices origin, $z$ and $w$ is equal to:\\
			\text{   }\hfill
			{(2021 - 4 Marks)}
			\begin{multicols}{4}
				\begin{enumerate}
						\item $\frac{1}{2}$
						\item 4
						\item 2
						\item $\frac{1}{4}$
				\end{enumerate}
			\end{multicols}
			
		}
    \item{
	
			The area bounded by the curve $4y^2 = x^2 \brak{4 - x}\brak{x - 2}$ is equal to:\\
			\text{   }\hfill
			{(2021 - 4 Marks)}
			\begin{multicols}{4}
				\begin{enumerate}
					\item $\frac{3\pi}{2}$
					\item $\frac{\pi}{16}$
					\item $\frac{\pi}{8}$
					\item $\frac{3\pi}{8}$
				\end{enumerate}
			\end{multicols}
			
		}
    \item{
        
            Define a relation $R$ over a class of $n \times n$ real matrices $A$ and $B$ as $A R B$ if  there exists a non-singular matrix $P$ such that $PAP^{-1} = B$. Then which of the following is true?\\ \text{ }
             \hfill
              {(2021 - 4 Marks)}
	              	\begin{enumerate}
	              		\item R is reflexive, symmetric but not transitive
	              		\item R is symmetric, transitive but not reflexive
	              		\item R is an equivalence relation
	              		\item R is reflexive, transitive but not symmetric
	              	\end{enumerate}
        
        }
    \end{enumerate}
\end{document}



