\let\negmedspace\undefined
\let\negthickspace\undefined
\documentclass[journal]{IEEEtran}
\usepackage[a5paper, margin=10mm, onecolumn]{geometry}
%\usepackage{lmodern} % Ensure lmodern is loaded for pdflatex
\usepackage{tfrupee} % Include tfrupee package

\setlength{\headheight}{1cm} % Set the height of the header box
\setlength{\headsep}{0mm}     % Set the distance between the header box and the top of the text

\usepackage{gvv-book}
\usepackage{gvv}
\usepackage{cite}
\usepackage{amsmath,amssymb,amsfonts,amsthm}
\usepackage{algorithmic}
\usepackage{graphicx}
\usepackage{textcomp}
\usepackage{xcolor}
\usepackage{txfonts}
\usepackage{listings}
\usepackage{enumitem}
\usepackage{mathtools}
\usepackage{gensymb}
\usepackage{comment}
\usepackage[breaklinks=true]{hyperref}
\usepackage{tkz-euclide} 
\usepackage{listings}
% \usepackage{gvv}                                        
\def\inputGnumericTable{}                                 
\usepackage[latin1]{inputenc}                                
\usepackage{color}                                            
\usepackage{array}                                            
\usepackage{longtable}                                       
\usepackage{calc}                                             
\usepackage{multirow}                                         
\usepackage{hhline}                                           
\usepackage{ifthen}                                           
\usepackage{lscape}
\begin{document}

\bibliographystyle{IEEEtran}
\vspace{3cm}

\title{1.1.5.33}
\author{EE24BTECH11059 - Yellanki Siddhanth
}
% \maketitle
% \newpage
% \bigskip
{\let\newpage\relax\maketitle}

\renewcommand{\thefigure}{\theenumi}
\renewcommand{\thetable}{\theenumi}
\setlength{\intextsep}{10pt} % Space between text and floats


\numberwithin{equation}{enumi}
\numberwithin{figure}{enumi}
\renewcommand{\thetable}{\theenumi}


\textbf{Question}:\\

The vectors $\lambda\hat{i} + \hat{j} +2\hat{k}$, $\hat{i} + \lambda\hat{j} - \hat{k}$ and $2\hat{i} - \hat{j} +\lambda\hat{k}$  are coplanar if $\lambda = $
\\ \textbf{Solution: }\\
    \begin{table}[h!]    
      \centering
      \begin{tabular}[12pt]{ |c| c| c|}
    \hline
    \textbf{Variable} & \textbf{Description} & \textbf{Value}\\
	\hline
	$\vec{n}$ &Normal of Directrix& $\myvec{1 \\ -1} $\\
	\hline
	$\vec{c}$ & c of Directrix& $ 3$\\
	\hline
	$\vec{e}$ & Eccentricity of conic & $\frac{1}{2}$\\
	\hline
	$\vec{F}$ & Focus of conic &  $\myvec{1 \\ -1}$ \\
	\hline
\end{tabular}

      \caption{}
    \end{table}\\
The rank of a matrix $M$ is the maximum number of linearly independent rows or columns in the matrix. If the rank of the matrix $M$ is 2, it means that there are only two linearly independent vectors in the set, and hence, the three vectors are coplanar (since the third vector is a linear combination of the first two).
    \begin{align}
        Rank(M) = 2\label{eq1.1.6.19.1}
    \end{align}
Equivalently,
    \begin{align}
        |M| = 0 \label{eq1.1.6.19.2}
    \end{align}
    \begin{align}
        |M| = \begin{vmatrix}\lambda & 1 & 2 \\ 1 & \lambda & -1 \\ 2 & -1 & \lambda\end{vmatrix}=0\label{eq1.1.6.19.3}
    \end{align}
    \begin{align}
        \lambda(\lambda^2 - 1) - 1(\lambda+2) + 2(-1-2\lambda) = 0  \label{eq1.1.6.19.4}
    \end{align}
    \begin{align}
        \lambda^3 - 6\lambda - 4 = 0  \label{eq1.1.6.19.5}
    \end{align}
    \begin{align}
        (\lambda + 2)(\lambda^2 - 2\lambda - 2) = 0  \label{eq1.1.6.19.6}
    \end{align}
Therefore,
    \begin{align}
         \lambda = -2 \text{ or } \lambda = 1 \pm \sqrt{3}\label{eq1.1.6.19.6}
    \end{align}
    
    \begin{figure}[h]
        \centering
       \includegraphics[width=0.7\linewidth]{figs/fig1.png}
       \caption{}
       \label{graph}
    \end{figure}
    \begin{figure}[h]
        \centering
       \includegraphics[width=0.7\linewidth]{figs/fig2.png}
       \caption{}
       \label{graph}
    \end{figure}
    \begin{figure}[h]
        \centering
       \includegraphics[width=0.7\linewidth]{figs/fig3.png}
       \caption{}
       \label{graph}
    \end{figure}    
\end{document}  




