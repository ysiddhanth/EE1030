\let\negmedspace\undefined
\let\negthickspace\undefined
\documentclass[journal]{IEEEtran}
\usepackage[a5paper, margin=10mm, onecolumn]{geometry}
%\usepackage{lmodern} % Ensure lmodern is loaded for pdflatex
\usepackage{tfrupee} % Include tfrupee package

\setlength{\headheight}{1cm} % Set the height of the header box
\setlength{\headsep}{0mm}     % Set the distance between the header box and the top of the text

\usepackage{gvv-book}
\usepackage{gvv}
\usepackage{cite}
\usepackage{amsmath,amssymb,amsfonts,amsthm}
\usepackage{algorithmic}
\usepackage{graphicx}
\usepackage{textcomp}
\usepackage{xcolor}
\usepackage{txfonts}
\usepackage{listings}
\usepackage{enumitem}
\usepackage{mathtools}
\usepackage{gensymb}
\usepackage{comment}
\usepackage[breaklinks=true]{hyperref}
\usepackage{tkz-euclide} 
\usepackage{listings}
% \usepackage{gvv}                                        
\def\inputGnumericTable{}                                 
\usepackage[latin1]{inputenc}                                
\usepackage{color}                                            
\usepackage{array}                                            
\usepackage{longtable}                                       
\usepackage{calc}                                             
\usepackage{multirow}                                         
\usepackage{hhline}                                           
\usepackage{ifthen}                                           
\usepackage{lscape}
\begin{document}

\bibliographystyle{IEEEtran}
\vspace{3cm}

\title{1.1.6.13}
\author{EE24BTECH11059 - Yellanki Siddhanth
}
% \maketitle
% \newpage
% \bigskip
{\let\newpage\relax\maketitle}

\renewcommand{\thefigure}{\theenumi}
\renewcommand{\thetable}{\theenumi}
\setlength{\intextsep}{10pt} % Space between text and floats


\numberwithin{equation}{enumi}
\numberwithin{figure}{enumi}
\renewcommand{\thetable}{\theenumi}


\textbf{Question}:\\
The points $\brak{0,5}$ , $\brak{0,-9}$ and $\brak{3,6}$ are collinear
\\ \textbf{Solution: }\\
    \begin{table}[h!]    
      \centering
      \begin{tabular}[12pt]{ |c| c| c|}
    \hline
    \textbf{Variable} & \textbf{Description} & \textbf{Value}\\
	\hline
	$\vec{n}$ &Normal of Directrix& $\myvec{1 \\ -1} $\\
	\hline
	$\vec{c}$ & c of Directrix& $ 3$\\
	\hline
	$\vec{e}$ & Eccentricity of conic & $\frac{1}{2}$\\
	\hline
	$\vec{F}$ & Focus of conic &  $\myvec{1 \\ -1}$ \\
	\hline
\end{tabular}

      \caption{}
    \end{table}\\
The rank of a matrix $M$ is 1, then the matrix is collinear. 
    \begin{align}
        Rank\brak{M} = 1\label{eq1.1.6.22.1}
    \end{align}
Computing matrix $M$
    \begin{align}
        M = \myvec{0 & 3\\ -14 & 1}\label{eq1.1.6.22.2}
    \end{align}
Clearly we can conclude that the rank of matrix $M$ is $\neq$ 1\\ $\therefore$ $A,B,C$ are not collinear.\\

\textbf{Alternate Solution}:
If said points $A,B,C$ are collinear, then the area of $\triangle ABC$ is 0.
    \begin{align}
        Area = \frac{1}{2}\abs{\abs{\brak{A-B}\times\brak{A-C}}}\label{eq1.1.6.22.3}
    \end{align}
    \begin{align}
        Area = \frac{1}{2}\abs{\abs{\myvec{0\\14}\times\myvec{-3\\-1}}}\label{eq1.1.6.22.3}
    \end{align}
    \begin{align}
        Area = \frac{1}{2}\abs{\abs{-42}} = 21 sq.units\label{eq1.1.6.22.3}
    \end{align}
Since $Area \neq 0$, the points $A,B,C$ are not collinear.
    \begin{figure}[h]
        \centering
       \includegraphics[width=0.7\linewidth]{figs/fig1.png}
       \caption{}
       \label{graph}
    \end{figure}



\end{document}  







