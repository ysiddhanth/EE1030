\let\negmedspace\undefined
\let\negthickspace\undefined
\documentclass[journal,12pt,twocolumn]{IEEEtran}
\usepackage{cite}
\usepackage{amsmath,amssymb,amsfonts,amsthm}
\usepackage{algorithmic}
\usepackage{graphicx}
\usepackage{textcomp}
\usepackage{xcolor}
\usepackage{txfonts}
\usepackage{listings}
\usepackage{enumitem}
\usepackage{mathtools}
\usepackage{gensymb}
\usepackage{comment}
\usepackage[breaklinks=true]{hyperref}
\usepackage{tkz-euclide} 
\usepackage{listings}
\usepackage{gvv}  
\usepackage{tikz}
\usepackage{circuitikz} 
\usepackage{caption}
\def\inputGnumericTable{}              
\usepackage[latin1]{inputenc}                   
\usepackage{color}                    
\usepackage{array}                     
\usepackage{longtable}                 
\usepackage{calc}                     \usepackage{multirow}                  
\usepackage{hhline}                    
\usepackage{ifthen}                    
\usepackage{lscape}
\usepackage{amsmath}
\newtheorem{theorem}{Theorem}[section]
\newtheorem{problem}{Problem}
\newtheorem{proposition}{Proposition}[section]
\newtheorem{lemma}{Lemma}[section]
\newtheorem{corollary}[theorem]{Corollary}
\newtheorem{example}{Example}[section]
\newtheorem{definition}[problem]{Definition}
\newcommand{\BEQA}{\begin{eqnarray}}
\newcommand{\EEQA}{\end{eqnarray}}
\newcommand{\define}{\stackrel{\triangle}{=}}
\theoremstyle{remark}
\newtheorem{rem}{Remark}

%\bibliographystyle{ieeetr}
\begin{document}
%

\bibliographystyle{IEEEtran}


%code by ysiddhanth 

\title{
%	\logo{
ASSIGNMENT 1: D16(7 TO 22)

\large{EE1030 : Matrix Theory}

Indian Institute of Technology Hyderabad
%	}
}
\author{Yellanki Siddhanth

(EE24BTECH11059)
}	





\maketitle

\newpage



\bigskip

\renewcommand{\thefigure}{\theenumi}
\renewcommand{\thetable}{\theenumi}
 
    
        {D: MCQs With One Or More Than One Correct.}
    
    \begin{enumerate}[start = 7]
    \item{
            If $f(x) = \displaystyle \frac{x^2-1}{x^2+1}$, for every real number $x$, then the minimum value of $f$
            \begin{flushright}
                {(1998 - 2 Marks)}
            \end{flushright}
            \begin{enumerate}
                \item does not exist because f is unbounded
                \item is not attained even though f is bounded
                \item is equal to 1
                \item is equal to -1
            \end{enumerate} }
    %code by ysiddhanth 
    \item{
            The number of values of the x where the function $f(x)=\cos x + \cos(\sqrt{2}x)$ attains its maximum is
            \begin{flushright}
                {(1998 - 2 Marks)}
            \end{flushright}
            \begin{enumerate}
                \item 0
                \item 1
                \item 2
                \item infinite
            \end{enumerate}
        }
    \item{
     
            The function $f(x)= \int_{-1}^x t(e^t-1)(t-1)(t-2)^3(t-3)^5dt$ has a local minimum at $x=$
            \begin{flushright}
                {(1999 - 3 Marks)}
            \end{flushright}
            \begin{enumerate}
                \item 0 
                \item 1
                \item 2
                \item 3
            \end{enumerate}
        
        }
    \item{
        
            $f(x)$ is a cubic polynomial with  $f(2)=18$ and $f(1)=-1$. Also $f(x)$ has a local maxima at $x=-1$ and $f'(x)$ has a local minima at $x=0$, then
            \begin{flushright}
                {(2006 - 5M, -1)}
            \end{flushright}
            \begin{enumerate}
                \item the distance between (-1,2) and ($a,f(a)$), where $x=a$ is the point of local minima is $2\sqrt{5}$
                \item $f(x)$ is increasing for $x$ $\in [1,2\sqrt{5}$]
                \item $f(x)$ has a local minima at $x$=1
                \item the value of $f(0)=15$
            \end{enumerate}
        
        }
    \item{
        
            $f(x) = $    
            $\begin{dcases}%code by ysiddhanth 
                e^x, & 0 \leq x\leq 1 \\
                2-e^{x-1}, & 1 < x\leq 2 \\
                x-e & 2 < x\leq 3 \\
            \end{dcases}$ and $g(x) = \int_0^{x}f(t)dt, x \in[1,3]$ then g(x) has
            \begin{flushright}
                {(2006 - 5M, -1)}
            \end{flushright}
            \begin{enumerate}
                \item has local maxima at $x = 1+ \ln{2}$ and local minima at $x = e$
                \item has local maxima at $x = 1$ and local minima at $x = 2$
                \item no local maxima
                \item no local minima
            \end{enumerate}
        
        }
    \item{
        
            For the function $$f(x) = x\cos{\frac{1}{x}}, x\geq1$$
            \begin{flushright}
                {(2009)}
            \end{flushright}
            \begin{enumerate}
                \item for at least one $x$ in the interval [1,$\infty$), $f(x+2)-f(x)<2$
                \item $\lim_{x\to\infty} f'(x) = 1$
                \item for all $x$ in the interval [1,$\infty$], $f(x+2)-f(x)>2$
                \item $f'(x)$ is strictly decreasing in the interval [1,$\infty$)
            \end{enumerate}
        
        }
    \item{
        
            If $f(x)= \int_{0}^x e^{t^2}(t-2)(t-3)dt$ for all $x \in(0,\infty)$, then
            \begin{flushright}
                {(2012)}
            \end{flushright}
            \begin{enumerate}
                \item $f$ has a local maximum at $x=2$ 
                \item $f$ is decreasing on (2,3)
                \item there exists some c$\in(0,\infty)$, such that $f''(c)=0$
                \item $f$ has a local minimum at $x=3$
            \end{enumerate}
        
        }
        \newpage
    \item{
    
        
            A rectangular sheet of fixed perimeter with sides having their lengths in the ratio 8: 15 is converted into an open rectangular box by folding after removing squares of equal area from all four corners. If the total area of removed squares is 100, the resulting box has maximum volume. Then the lengths of the sides of the rectangular sheet are
            \begin{flushright}
                {(JEE Adv. 2013)}
            \end{flushright}
            \begin{enumerate}
                \item 24
                \item 32
                \item 45
                \item 60
            \end{enumerate}
        %code by ysiddhanth 
        
        }
    \item{
        
            Let $f: (0,\infty)\rightarrow R$ be given by $f(x) = \int_{\frac{1}{x}}^x e^{-(t + \frac{1}{t})\ \frac{dt}{t}}$. Then
            \begin{flushright}
                {(JEE Adv. 2014)}
            \end{flushright}
            \begin{enumerate}
                \item $f(x)$ is monotonically increasing on [1,$\infty$)
                \item $f(x)$ is monotonically decreasing on (0,1)
                \item $f(x) + f(\frac{1}{x}) = 0$, for all $x \in (0,\infty)$
                \item $f(2^x)$ is an odd function of $x$ on $R$ 
            \end{enumerate}
        
        }
    \item{
        
            Let $f,g$: [-1,2]$\rightarrow R$ be continuous functions which are twice differentiable on the interval (-1,2). Let the values of $f$ and $g$ at the points -1,0 and 2 be as given in the following table:
            \begin{center}
            \begin{tabular}{|c|c|c|c|} 
              \hline
               & $x=-1$ & $x = 0$ & $x = 2$ \\ 
              \hline
               $f(x)$& 3 & 6 & 0\\ 
              \hline
               $g(x)$ & 0 & 1 & -1\\ 
              \hline
            \end{tabular}
            \end{center}  
            In each of the intervals (-1,0) and (0,2) the function $(f-3g)''$ never vanishes. Then the correct statement(s) is(are)
            \begin{flushright}
                {(JEE Adv. 2015)}
            \end{flushright}
            \begin{enumerate}
                \item $f'(x) - 3g'(x) = 0$ has exactly three solutions in (-1,0)$\cup$(0,2)
                \item $f'(x) - 3g'(x) = 0$ has exactly one solution in (-1,0)
                \item $f'(x) - 3g'(x) = 0$ has exactly one solution in (0,2)
                \item $f'(x) - 3g'(x) = 0$ has exactly two solutions in (-1,0)
            \end{enumerate}
        
        }
    \item{
        
    %code by ysiddhanth 
            Let $f: R\rightarrow (0,\infty)$ and $g: R\rightarrow R$ be twice differentiable functions such that $f''$ and $g''$ are continuous functions on $R$. Suppose $f'(2) = g(2)=0, f''(2) \neq0$ and $g'(2)\neq0$.\\[6pt] If $\lim_{x\to2} \displaystyle \frac{f(x)g(x)}{f'(x)g'(x)} = 1$, then
            \begin{flushright}
                {(JEE Adv. 2016)}
            \end{flushright}
            \begin{enumerate}
                \item $f$ has a local minimum at $x=2$
                \item $f$ has a local maximum at $x=2$
                \item $f''(2)>f(2)$
                \item $f(x) - f''(x) = 0$ for at least one $x\in R$
            \end{enumerate}
        
        }
    \item{
        
            If $f: R\rightarrow R$ a differentiable function such that $f'(x)>2f(x)$ for all $x \in R$ and $f(0) = 1$, then
            \begin{flushright}
                {(JEE Adv. 2017)}
            \end{flushright}
            \begin{enumerate}
                \item $f(x)$ is increasing in $(0,\infty)$
                \item $f(x)$ is decreasing in $(0,\infty)$
                \item $f(x)>e^{2x}$ in $(0,\infty)$
                \item $f'(x)<e^{2x}$ in $(0,\infty)$
            \end{enumerate}
        
        }
    \item{
        %code by ysiddhanth 
        
            If $f(x) = $
            \begin{tabular}{|c c c|} 
    
               $\cos 2x$& $\cos 2x$ & $\sin 2x$ \\ 
    
               $- \cos x$& $\cos x$ & $\sin x$\\ 
    
               $\sin x$ & $\sin x$ & $\cos x$\\ 
    
            \end{tabular} 
            , then
            \begin{flushright}
                {(JEE Adv. 2017)}
            \end{flushright}
            \begin{enumerate}
                \item $f'(x) = 0$ at exactly three points in $(\pi,\pi)$
                \item $f'(x) = 0$ at more than three points in $(\pi,\pi)$
                \item $f(x)$ attains its maximum at $x=0$
                \item $f(x)$ attains its minimum at $x=0$
            \end{enumerate}
        
        }
    \item{
        
            Define the collection $\{E_1, E_2, E_3, .....\}$ of ellipses and $\{R_1, R_2, R_3, .....\}$ of rectangles as follows:\\[6pt]
            $E_1 : \displaystyle\frac{x^2}{9}+\displaystyle \frac{y^2}{4} = 1$;\\[6pt]
            $R_1$: rectangle of largest area, with sides parallel to the axes, inscribes in $E_1$;\\[6pt]
            $E_n : $ ellipse $\displaystyle\frac{x^2}{a_n^2}+\displaystyle \frac{y^2}{b_n^2} = 1$ of largest area inscribed in $R_{n-1}, n>1;$\\[6pt]
            $R_n$: rectangle of largest area, with sides parallel to the axes, inscribes in $E_n$;\\ Then which of the following options is/are correct?
            \begin{flushright}
                {(JEE Adv. 2019)}
            \end{flushright}
            \begin{enumerate}
                \item The eccentricities of $E_{18}$ and $E_{19}$ are not equal
                \item Length of the latus rectum of $E_{9}$ is $\displaystyle\frac{1}{6}$
                \item $\sum_{n=1}^N$ (area of $R_n$) $<$ 24, for each positive integer N
                \item The distance of a focus from the centre in $E_9$ is $\frac{\sqrt{5}}{32}$
            \end{enumerate}
        
        }
        \newpage
    \item{
        
            Let $f: R\rightarrow R$ be given by $f(x) = (x-1)(x-2)(x-5)$.\\[6pt] 
            Define $F(x) = \displaystyle\int_0^x f(t)dt, x>0$.\\[3pt]
            Then which of the following options is/are correct?
            \begin{flushright}
                {(JEE Adv. 2019)}
            \end{flushright}
            \begin{enumerate}
                \item $F$ has a local maximum at $x=2$
                \item $F$ has a local minimum at $x=1$
                \item $F$ has two local maxima and one local minimum in $(0,\infty)$
                \item $F(x) = 0$ for all $x \in (0,\infty)$
            \end{enumerate}
        
        }
    \item{
        
            Let $f(x) = \displaystyle\frac{\sin(\pi x)}{x^2}, x>0$.\\[3pt]
            Let $x_1 < x_2 < x_3 < .......... <x_n <......$ be all the points of local maximum of $f$ and $y_1 < y_2 < y_3 < .......... <y_n <......$ be all the points of local minimum of $f$.\\
            Then which of the following options is/are correct?
            \begin{flushright}
                {(JEE Adv. 2019)}
            \end{flushright}
        %code by ysiddhanth 
            \begin{enumerate}
                \item $x_{n+1} - x_{n} > 2$ 
                \item $x_{n} \in (2n, 2n+\frac{1}{2})$ for every n
                \item $|x_{n} - y_{n}|> 1 $ for every n
                \item $x_{1} < y_{1}$
            \end{enumerate}
        
        }
    \end{enumerate}
\end{document}
%code by ysiddhanth 
